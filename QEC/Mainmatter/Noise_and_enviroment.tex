\section{Quantum errors: Noise and Decoherence}
The Schrödinger equation
$$
i \hbar \frac{d|\psi\rangle}{d t}=H|\psi\rangle
$$
describes the evolution of quantum systems in isolation, where $|\psi\rangle$ is the state vector. These closed systems have a well-defined Hamiltonian operator $H$, which gives complete information about how these systems evolve. The resulting evolution is unitary: the evolution of the state is given by a linear map $|\psi(0)\rangle \rightarrow|\psi(t)\rangle=U|\psi(0)\rangle$ where $U^{\dagger} U=$ $U U^{\dagger}=I$. Note that these Hamiltonians may "come from outside" the system; for instance, we can turn external fields on and off, shine lasers, etc. What makes a quantum system closed is that it doesn't act back on the external world. The external fields, lasers, etc., can all be treated as classical potentials.

The unfortunate reality is that this idealization is a fiction. All real quantum systems interact with the outside world, at least weakly; and the existence of interactions, which allow us to manipulate a system, also allow the system to interact with the external environment. This environmental interaction is called decoherence.

One can model the dynamics of a register of qubits with its surroundings. We imagine the system immersed into its environment (often called bath) and the whole (quantum register plus environment) as a closed system described in a general way by the following Hamiltonian:
$$
H=H_{S} \otimes I_{B}+I_{S} \otimes H_{B}+H_{\mathcal{I}}
$$
where $H_{S}\left(H_{B}\right)$ (the system (bath) Hamiltonian) acts on the system (bath) Hilbert space $\mathcal{H}_{S}$ $\left(\mathcal{H}_{B}\right), I_{S}\left(I_{B}\right)$ is the identity operator on the system (bath) Hilbert space, and $H_{I}$, which acts on both the system and bath Hilbert spaces $\mathcal{H}_{S} \otimes \mathcal{H}_{B}$, is the interaction Hamiltonian containing all the nontrivial couplings between system and bath. In general $H_{\mathcal{I}}$ can be written as a sum of operators which act separately on the system $\left(S_{\alpha}\right.$) and on the bath $\left(B_{\alpha}\right.$):
$$
H_{I}=\sum_{\alpha} S_{\alpha} \otimes B_{\alpha}
$$
(Note that this decomposition is not necessarily unique). In the absence of an interaction Hamiltonian $\left(H_{\mathcal{I}}=0\right)$, the evolution of the system and the bath are separately unitary: $\mathbf{U}(t)=\exp (-i \frac{H}{\hbar} t)=\exp \left(-i \frac{H_{S}}{\hbar} t\right) \otimes \exp \left(-i \frac{H_{B}}{\hbar} t\right)$ . Information that has been encoded (mapped) into states of the system Hilbert space remains encoded in the system Hilbert space if $H_{\mathcal{I}}=0$.
 However in the case when the interaction Hamiltonian contains nontrivial couplings between the system and the environment ($H_{\mathcal{I}}\neq 0$),decoherence happens. %information that has been encoded over the system Hilbert space does not remain encoded over solely the system Hilbert space but spreads out instead into the combined system and bath Hilbert space as the time evolution proceeds.

Two things happen in decoherence. First, random influences from the outside can perturb the system's evolution, as if some random Hamiltonian was turned on, in addition to the usual Hamiltonian. Second, the interaction between the system and environment can cause information about the system to leak into the environment. This information leakage leaves the system correlated with the environment. In fact, the information that was carried by the system before, then a certain quantity is spread out also in the environment, the No-hiding theorem show how this property works. The effect on the system is as if unwanted measurements have been performed (without, in general, our knowing the measurement results).
In fact, these two processes generally both occur, and the practical effects of them often look similar. Indeed, in quantum mechanics there is no sharp distinction between them. If decoherence persists long enough, it is possible for all information about the original state of the system to be lost. In the shorter term, decoherence can destroy quantum effects such as interference and entanglement (on which quantum information processing depends).

In general, we can study the evolution of an isolate quantum systems subjected to decoherence and other sources of noise using quantum states, but as already said in the previous section, the state vector approach does not immediately allow for a formalisation of ignorance or missing knowledge, hence a more general description can be done using the density matrix formalism. This allows for the possibility to study mixed states (i.e., contains uncertainties that represent missing information) rather than only pure (isolated and perfectly known).
Then, once the initial state is represented by density operator we can find a function with certain properties that describes the evolution of the density operator.
Maps that represent this evolution must preserve these properties: they are completely positive, trace-preserving and map density matrix into density matrix.
A suitable ensemble of maps that satisfy this properties are called CPTP maps. These maps can be written as
$$
\rho' = \mathcal{E}(\rho) \rightarrow \sum_{\mu} K_{\mu} \rho K_{\mu}^{\dagger} \quad \text { with } \sum_{\mu} K_{\mu}^{\dagger} K_{\mu}=I
$$
where the $K_{\mu}$ are $N \times N$ matrices ($N$ being the dimension of the Hilbert space) and are called Kraus operators. 
In general, the Kraus decomposition of a CPTP map is not unique, but one can approximately think of the map as the state $|\psi\rangle$ being multiplied by one of the operators $K_{\mu}$ chosen at random with probability $p_{\mu}=\bra{\psi} K_{\mu}^{\dagger} K_{\mu}\ket{\psi} .$ Since one does not know which operator has multiplied the state, one uses a mixture of all of them.

In a similar way, if an unknown influence is applied to the quantum system from the outside, we can model that as a set of unitaries $\left\{A_{\mu}\right\}$ that occur with respective probabilities $\left\{p_{\mu}\right\}$. Here, again, one would describe the state of the system as a mixture of all possible evolved states:
$$
\rho \rightarrow \rho^{\prime}=\sum_{\mu} p_{\mu} A_{\mu} \rho A_{\mu}^{\dagger}, \quad \sum_{\mu} p_{\mu}=1
$$
In this case again we have a CPTP map, and we can define the Kraus operators to be $K_{\mu} \equiv \sqrt{p_{\mu}} A_{\mu} .$ Note that the randomness in the unitary evolution need not be due to outside influence:
it could also be from uncertainty of the Hamiltonian, due to imperfect control of the system or any other reason. CPTP maps give a unified description of all possible sources of Markovian noise\footnote{ Markovian dynamics means is a limit or approximation in which the recent details don't matter, that is, the environment is "memoryless" and does not contain information about the history of the system. This is good because it allows to write an equation for $\rho(t)$ that depends only on $\rho(t)$, and not on, say, $\rho\left(t-t^{\prime}\right)$. This is called "Time local" equation and is much easier to handle.}, and in quantum information science one does not usually make a sharp distinction between different noise sources. 




\section{Fidelity}
In comunication problems we would know how much information is preserved by some process. We would like to compare the initial message and the final message after the effect of noise. 
A quantity that can represent how much two states are similar is the fidelity. 
Hence, it is possible to use the fidelity as an appropriate measure of the quality of a recovered code.
A clever way to define the fidelity is to use the desity matrix formalism.
Fidelity is the overlap between the final state $\rho_f$ of a system $\rho$ and the original state $\ket{\psi}$.
If the combined operator consisting of an interaction with the environment followed by a recovery operation is given by $\mathcal{A}=\left\{A_{0}=\mathcal{R}_0E_0, \ldots\right\}$, then the fidelity is
$$
\mathcal{F}\left(\rho_f,\ket{\psi}_i\right)=\bra{\psi_i}\rho_{f}\ket{\psi_i}=\sum_{a}\bra{\psi_i}A_{a}\ket{\psi_i}\bra{\psi_i}A_{a}^{\dagger}\ket{\psi_i} .
$$
It gives the probability that the final state would pass a test checking whether it agrees with the initial state. As we are thinking of encoding arbitrary states, we do not know in advance the state that will be used. We therefore use the minimum fidelity (that is the worst case fidelity)
\begin{equation}
\mathcal{F}_{\min }=\min _{|\psi\rangle}\left\langle\psi\left|\rho_{f}\right| \psi\right\rangle .
\end{equation}
The best quantum code maximizes $\mathcal{F}_{\min }$. 


A quantum communication channel can be use to transmit the state $\ket{\psi}$ from one location to another. No channel is ever perfect, so the action of the channel is described by a quantum operation $\mathcal{E}$ on $\rho=\ket{\psi}\bra{\psi}$. A way to quantify how likely I get $\ket{\psi}$ at the end of the channel is the fidelity. 
Let's take an example of a noisy channel: the depolarising channel. This channel leaves the qubit untouched with probability $1-p$ and with probability $\frac{p}{3}$ one of these 3 error $\{X,Y,Z\}$ can act on the qubit: 
\begin{equation}
    \rho'=\mathcal{D}(\rho) = (1-p)\rho + \frac{p}{3}\left(X\rho X +Y \rho Y+Z \rho Z\right)
\end{equation}
We can write the expression differently: 
$$
\begin{aligned}
\rho^{\prime} &=\frac{p}{3}(X \rho X+Y \rho Y+Z \rho Z)+(1-p) \rho \\
&=\frac{p}{3}(\rho+X \rho X+Y \rho Y+Z \rho Z)+\left(1-\frac{4 p}{3}\right) \rho \\
\end{aligned}
$$
We can simplify the expression more, we can use the following relation\footnote{It is straightforward to calculate that $\mathcal{E}(I)=I$ and $\mathcal{E}(X)=\mathcal{E}(Y)=\mathcal{E}(Z)=0$. However, $I, X, Y, Z$ form a basis of $2\times2$ matrices, so any $\rho$ can be written as $\rho=a I+b X+c Y+d Z$ and then $\mathcal{E}(\rho)=a I$. If $\rho$ is a density matrix then $a=\frac{1}{2}$ and $\mathcal{E}(\rho)=I / 2$. This means that $\mathcal{E}$ maps a pure state into a mixed state, for example $\mathcal{E}(\ket{0}\bra{0}) =\frac{1}{2}(\ket{0}\bra{0} + \ket{1}\bra{1})$}: 
$$
\mathcal{E}(\rho)=\frac{1}{4}(\rho+X \rho X+Y \rho Y+Z \rho Z)=\frac{I}{2}
$$
We can use this result to extract the term proportional to the identity: 
$$
\begin{aligned}
&=\frac{4 p}{3} \frac{I}{2}+\left(1-\frac{4 p}{3}\right) \rho \\
&=\lambda \frac{I}{2}+(1-\lambda) \rho
\end{aligned}
$$
The $p \in[0,1]$ is the Pauli error probability and $\lambda=\frac{4 p}{3} \in\left[0, \frac{4}{3}\right]$ is the depolarization parameter.These two parameters are different because the maximally mixed state $I / 2$ and $(X \rho X+Y \rho Y+Z \rho Z) / 3$ are different states.
From this we can calculate the fidelity between the initial and final mixed state: 
\begin{align*}
    \mathcal{F}(\ket{\psi},\rho') = \text{min}_{\ket{\psi}} &= \bra{\psi}\rho'\ket{\psi} \\
    &= \bra{\psi}(\lambda \frac{I}{2}+(1-\lambda) \ket{\psi}\bra{\psi})\ket{\psi} \\
    &= \frac{\lambda}{2} + (1-\lambda) = 1-\frac{2}{3}p 
\end{align*}
This result agrees well with our intuition the higher the probability $p$ of depolarizing, the lower the fidelity of the final state with the initial state. Provided $p$ is very small the fidelity is close to one, and the state $\mathcal{E}(\rho)$ is practically indistinguishable from the initial state. More examples can be done using different channels in the same way.


So far, we looked only when the state sent into the channel is not entangled, but in quantum error correcting code in order to protect a qubit we have to entangle it. 
The states to be protected involve a subset of entangled qubits. This means that in discussions of fidelity and error, the whole state, not just the component being protected, must be considered.
The worst case fidelity for such states is referred to as the entangled state fidelity to distinguish it from the pure state fidelity introduced earlier.
If the pure state fidelity after recovery of the coded subsystem is one, then the entangled state fidelity is one also; it does not matter if the state is pure or if it is entangled with other systems. This observation is invalid if we have imperfect fidelity.
%We can also characterize the channel by introducing a reference qubit $R$ and describing how a maximally-entangled state of the two qubits $RA$ evolves, when the channel acts only on $A$. There are four mutually orthogonal maximally entangled states, They also form a basis and are called Bell's states. 
%If the initial state is $\left|\phi^{+}\right\rangle_{R A}$, then when the depolarizing channel acts on qubit $A$, the entangled state evolves as
% $$
% \left|\phi^{+}\right\rangle\left\langle\phi^{+}|\mapsto(1-p)| \phi^{+}\right\rangle\left\langle\phi^{+}\right|+\frac{p}{3}\left(\left|\psi^{+}\right\rangle\left\langle\psi^{+}|+| \psi^{-}\right\rangle\left\langle\psi^{-}|+| \phi^{-}\right\rangle\left\langle\phi^{-}\right|\right) .
% $$
Consider a quantum system of combined two quantum subsystems labeled
as $A$ and $B$. The state of $A$ is assumed to be entangled in some way with the external world that we indicates with $B$. Suppose the joint system $AB$ initially is prepared in a general state $\rho_{i}=\ket{\psi_{AB}}\bra{\psi_{AB}}$. This is an entangled state, and we would like to know how well an entangled state is preserved after the action of noise.
Assume that only the subsystem $A$ is affected by the noisy channel with
some evolution described by a quantum operation $\mathcal{E} = \{E_a\}$, for example it can be a bit flip with probability $p_a$. While the subsystem $B$
is dynamically isolated. In this case, the overall dynamics of the joint system $AB$ is described by the quantum operation $\mathcal{E} \otimes I$, where $I$ here is the identity operator acting on the subsystem $B$. Thus the final state of the joint system is given by the density operator $\rho_{\mathrm{f}}= \mathcal{E}\left(\rho_{\mathrm{i}}\right) \otimes I $.
Such as for the isolated state, we can compute the fidelity for an entangled state $\mathcal{F}_e$ as well.
\begin{equation}
\mathcal{F}_{e}(\ket{\psi_{AB}},\rho_f)=\min _{\left|\psi_{AB}\right\rangle }\left\langle\psi_{AB}\left|\rho_f\right| \psi_{AB}\right\rangle
\label{eq:entfidel}
\end{equation}
The $\mathcal{F}_e$, in fact, takes its value in the interval $[0, 1]$, where values close to 1 are supposed to imply that the entanglement is well preserved and values close to $0$ indicate that the entanglement is mostly destroyed.
Using the operation sum representation the action of noise on the initial state is :
\begin{equation*}
\rho_f=(\mathcal{E}\otimes I)(\rho)=\sum_{a} E_a\ket{\psi_{AB}}\bra{\psi_{AB}} E_a^{\dagger}\end{equation*}
Then, we can substitute this in eq(\ref{eq:entfidel}): 
\begin{align*}
    \mathcal{F}_e&=\text{min}_{\psi_{AB}} \sum_a  \bra{\psi_{AB}}E_a\ket{\psi_{AB}}\bra{\psi_{AB}} E_a^{\dagger}\ket{\psi_{AB}} \\&= \text{min}_{\psi_{AB}} \sum_a  |\bra{\psi_{AB}}E_a\ket{\psi_{AB}}|^2
\end{align*}
Then, we can write the entangled state in the Schmidt basis as $\left|\psi_{AB}\right\rangle=\sum_{i} \sqrt{p_{i}}\left|\psi_{i}^{A}\right\rangle\left|\psi_{i}^{B}\right\rangle$:
\begin{align*}
    \bra{\psi_{AB}}E_a\ket{\psi_{AB}} &= \sum_{ij} \sqrt{p_i p_j} \braket{\psi_i^B|\psi_j^B} \braket{\psi_i^A|E_a|\psi_j^A}\\ 
    &= \sum_i p_i \braket{\psi_i^A|E_a|\psi_i^A}\\
    &= Tr(\rho E_a)
\end{align*}
Hence: 
\begin{equation}
    \mathcal{F}_e = \sum_a \left(Tr(\rho E_a)\right)^2
\end{equation}

Thus, for example, consider the interaction consisting of scalar multiples of the Pauli spin matrices,
$$
\mathcal{E}=\left\{\frac{1}{\sqrt{3}} X, \frac{1}{\sqrt{3}} Y, \frac{1}{\sqrt{3}} Z\right\} .
$$
We show that for this example, $F(\mathcal{E})=\frac{1}{3}$ (Fidelity for a not entangled state) and
$F_{e}(\mathcal{E}(\rho))=0 $  (entangled fidelity).

Consider the general state $|u\rangle=$ $\alpha|0\rangle+e^{i \theta} \beta|1\rangle$ with $\alpha$ and $\beta$ real, and $\alpha^{2}+\beta^{2}=1$. The fidelity $\mathcal{F}(\mathcal{E}(\rho), \ket{u})$ is obtained by the following expression
$$
\begin{aligned}
\mathcal{F}=\frac{1}{3}\left(\left|\left\langle u\left|X\right| u\right\rangle\right|^{2}\right.&\left.+\left|\left\langle u\left|Y\right| u\right\rangle\right|^{2}+\left|\left\langle u\left|Z\right| u\right\rangle\right|^{2}\right) \\
&=\frac{1}{3}\left((2 \alpha \beta \cos (\theta))^{2}+(2 \alpha \beta \sin (\theta))^{2}+\left(\alpha^{2}-\beta^{2}\right)^{2}\right)
\end{aligned}
$$
$$
\begin{array}{l}
=\frac{1}{3}\left(\left(\alpha^{2}+\beta^{2}\right)^{2}\right) \\
=\frac{1}{3} .
\end{array}
$$
Hence $F(\mathcal{E}(\rho),\ket{u})=\frac{1}{3}$. Then let us calculate the entangled fidelity considering as initial state the completely entangled state $|e\rangle=\frac{1}{\sqrt{2}}(|0\rangle|0\rangle+|1\rangle|1\rangle) .$
Then the entangled fidelity $F_{e}(\mathcal{E}(\rho),\ket{e})$ is given by \ref{eq:entfidel}, where we apply $\mathcal{E}$ only to the system A: 
$$
\begin{aligned}
 \sigma_{x} \otimes I|e\rangle &=\frac{1}{\sqrt{2}}(|0\rangle|1\rangle+|1\rangle|0\rangle) \\
\sigma_{y} \otimes I|e\rangle &=\frac{i}{\sqrt{2}}(|0\rangle|1\rangle-|1\rangle|0\rangle) \\
\sigma_{z} \otimes I|e\rangle &=\frac{i}{\sqrt{2}}(|0\rangle|0\rangle-|1\rangle|1\rangle)
\end{aligned}
$$
These states are all orthogonal to $|e\rangle$, hence $F_{e}(\mathcal{E}(\rho))=0$.